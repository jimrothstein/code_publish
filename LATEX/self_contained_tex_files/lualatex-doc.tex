% lualatex-doc: a guide to LuaLaTeX
%
% Written by Manuel Pégourié-Gonnard <mpg@elzevir.fr>, 2010.
%
% Distributed under the terms of the GNU free documentation licence:
%   http://www.gnu.org/licenses/fdl.html
% without any invariant section or cover text.

\documentclass{lltxdoc}

\title{A guide to \lualatex}
\author{Manuel Pégourié-Gonnard \email{mpg@elzevir.fr}}
\date{\today}

\begin{document}

\maketitle

\begin{abstract}
  This document is a map, or touristic guide, for the new world of
  \lualatex.\footnote{Though focusing on \lualatex, it includes relevant
    information about \luatex with the Plain format, too.} The intended
  audience ranges from complete newcomers (with a working knowledge of
  conventional \latex) to package developers. This guide is intended to be
  comprehensive in the following sense: it contains pointers to all relevant
  sources, gathers information that is otherwise scattered, and adds
  introductory material.

  Feedback and suggestions for improvement are most welcome. This document is
  work in progress; thanks for your comprehension and patience.
\end{abstract}

\vspace{\stretch{1}}
\setcounter{tocdepth}{2}
\listoftoc*{toc}
\vspace*{\stretch{2}}
\clearpage

\section{Introduction}\label{intro}

\subsection{Just what is \lualatex?}\label{what}

To answer this question, we need to mention a few details about the \tex world
that you may usually ignore: the difference between an \emph{engine} and a
\emph{format}. An engine is an actual computer program, while a format is a
set of macros executed by an engine, usually preloaded when the engine is
invoked with a particular name.

Actually, a format is more or less like a document class or a package, except
it is associated with a particular command name. Imagine there is a command
\cmd{latex-article} that would do the same as \cmd{latex}, except you wouldn't
need to say ©\documentclass{article}© at the beginning of your file.
Similarly, in current distributions, the command \cmd{pdflatex} is the same as
the command \cmd{pdftex} except that you don't need to put the instructions to
load \latex at the beginning of your source file. This is convenient, and
slightly more efficient too.

Formats are great because they implement powerful commands using the basic
tools an engine provides.  However the power of the format is sometimes
limited by the engine's tools set, so people started developing more
powerful engines so that other people can implement even more powerful formats
(or packages). The most famous engines now (except the original \tex) are
\pdftex, \xetex and \luatex.

To further complicate the picture, the original \tex engine produced only DVI
files, while its successors may (also) produce PDF file.  Each command in your
system corresponds to a particular engine with a particular format and a
particular output mode. The following table summarizes this: rows are indexed
by format, columns by engine, and in each cell the first line is the command
for this engine with this format in DVI mode, and the second for the same PDF
mode.

\begin{center}
  \newcommand*\cell [2] {%
    \parbox{6em}{\centering\leavevmode\color{code}\ttfamily
      \strut\maybe{#1} \\ \strut\maybe{#2}}}
  \makeatletter
  \newcommand*\maybe [1] {%
    \@ifmtarg{#1} {\textcolor{gray}{\normalfont (none)}} {#1}}
  \begin{tabular}{l|cccc}
                & \tex & \pdftex & \xetex & \luatex
    \\ \hline
    Plain
                &  \cell{tex}{}
                &  \cell{etex}{pdftex}
                &  \cell{}{xetex}
                &  \cell{dviluatex}{luatex}
    \\ \hline
    \latex
                &  \cell{}{}
                &  \cell{latex}{pdflatex}
                &  \cell{}{xelatex}
                &  \cell{dvilualatex}{lualatex}
    \\
  \end{tabular}
\end{center}

We can now answer the title question: \lualatex is the \luatex engine with the
\latex format. Well, this answer isn't very satisfying if you don't know what
\luatex and \latex are.

\medskip

As you probably know, \latex is the general framework
in which documents begin with ©\documentclass©, packages are loaded with
©\usepackage©, fonts are selected in a clever way (so that you can switch to
boldface while preserving italics), pages are build with complicated
algorithms
including support for headers, footers, footnotes, margin notes, floating
material, etc. This mostly doesn't change with \lualatex, but new and more
powerful packages are available to make parts of the system work in a better
way.

So, what's \luatex? Short version: the hottest \tex engine right now!
Long version: It is the designated successor of \pdftex and
includes all of its core features: direct generation of PDF files with support
for advanced PDF features and micro-typographic enhancements to \tex
typographic algorithms. The main new features of \luatex are:
\begin{enumerate}
  \item Native support of Unicode, the modern standard for character
    classification and encoding, supporting all characters in the world, from
    English to traditional Chinese through Arabic, including a lot of
    mathematical (or otherwise specialised) symbols.
  \item Inclusion of Lua as an embedded scripting language (see
    section~\ref{luaintex} for details).
  \item A lot of wonderful Lua libraries, including:
    \begin{itemize}
      \item ©fontloader©, supporting modern font formats such as
        TrueType and OpenType;
      \item ©font©, allowing advanced manipulation of the fonts
        from within the document;
      \item ©mplib©, an embedded version of the graphic program MetaPost;
      \item ©callback©, providing hooks into parts of the \tex engine that
        were previously inaccessible to the programmer;
      \item utility libraries for manipulating images, pdf files, etc.
    \end{itemize}
\end{enumerate}
Some of these features, such as Unicode support, directly impact all
documents, while others merely provide tools that package authors will use to
provide you with more powerful commands and other enhancements.

\subsection{Switching from \latex to \lualatex}\label{switch}

As the previous section explains, \lualatex is mostly like \latex, with a few
differences, and more powerful packages and tools available. Here we present
an absolute minimum you should know to produce a document with \lualatex, while
the rest of the document provides more details about the available packages.

There are only three differences:
\begin{enumerate}
  \item Don't load \pf{inputenc}, just encode your source in UTF-8.
  \item Don't load \pf{fontenc} nor \pf{textcomp}, but load \pf{fontspec}.
  \item \pf{babel} works with \lualatex{} but you can load \pf{polyglossia} instead.
  \item Don't use any package that changes the fonts, but use \pf{fontspec}'s
    commands instead.
\end{enumerate}
So, you only need to get familiar with \pf{fontspec}, which is
easy: select the main (serif) font with ©\setmainfont©, the sans serif font
with ©\setsansfont© and the mono-spaced (typewriter) font with ©\setmonofont©.
The argument to these commands is the human-friendly name of the font, for
example ©Latin Modern Roman© rather than ©ec-lmr10©. You probably want to use
©\defaultfontfeatures{Ligatures=TeX}© before these commands to keep the usual
\tex substitutions (such as ©---© for an em-dash) working.

The good news is: you can directly access any font on your operating system
(in addition to those of your \tex distribution) including TrueType and
OpenType fonts and have access to their most advanced features. It means it
is now easy to install for use with \lualatex any modern font you may download
or purchase from an editor and benefit from their full potential.

Now for the bad news: it is not always easy to get a list of all available
fonts. Under Windows with \texlive, the command-line tool \cmd{fc-list} lists
them all, but is not very friendly. Under Mac OS X, the \emph{Fontbook}
application lists the fonts of your system, but not those of your \tex
distribution. Same with \cmd{fc-list} on Linux. More bad news: it is not easy
to access your old fonts that way. Happily, more fonts are available in modern
formats every day (well, month or year, actually, if you count only good
fonts).

\emph{En passant}, let's mention that the content of this section so far also
holds for \xelatex, that is, \latex over \xetex. Indeed, \xetex shares two of
the essential features of \luatex: native Unicode and support for modern font
formats (but doesn't have the other features of \luatex; on the other hand, it
is more stable right now). Though their implementations concerning fonts are
vastly different, \pf{fontspec} manages to offer a mostly unified font
interface for both \xelatex and \lualatex.

\medskip

So, to benefit from the new features of \luatex, you must drop a few parts of
the old world, namely the fonts that are not available in a modern format (and
also the liberty to encode your source how you want, but UTF-8 is so much
superior that this one hardly counts). The package \pk{luainputenc} provides
various trade-offs that allow you to regain these parts\footnote{While the
  name suggests it is only about input encodings, the details of \latex's font
  encoding implementation imply this package is needed (and works) for old
  fonts too.} possibly at the expense of loosing real Unicode support.

That's all you need to know to start producing documents with \lualatex. I
recommend you have a look at the \pf{fontspec} manual and actually try to
compile a small document using funny fonts. You can then skim over the rest of
this document as you wish. Section~\ref{workornot} lists all the other
differences between conventional \latex and \lualatex that I'm aware of.

\subsection{A Lua-in-\tex primer}\label{luaintex}

Lua is a nice, small language, obviously less surprising and much easier to
learn than \tex as a programming language. The essential reference is the very
readable book \emph{Programming in Lua}, whose first edition is freely
\href{http://www.lua.org/pil/} {available online}. For a quick start, I
recommend you read chapters~1 to~5 and have a quick glance at part~3. Note
that all the libraries mentioned in chapter~3 are included in \luatex, but the
©os© library is restricted for security reasons.

Depending on your programming culture, you may be directly interested in the
rest of part~1 and part~2 which present more advanced features of the
language, but part~4 is useless in a \luatex context, unless of course you
want to hack \luatex itself. Finally, the \emph{Lua reference manual} is
\href{http://www.lua.org/manual/5.2/}{available online} and
comes with a handy index.

\medskip

Now, let's turn to Lua in \luatex. The main way to execute Lua code from the
\tex end is the ©\directlua© command, which takes arbitrary Lua code as an
argument. Conversely, you can pass information from Lua to \tex with
©tex.sprint©.\footnote{The name probably means ``string print'' as opposed to
  ``run very fast for a short period of time.''} For example,
\begin{Verbatim}
  the standard approximation $\pi = \directlua{tex.sprint(math.pi)}$
\end{Verbatim}
prints ``the standard  approximation $\pi = \directlua{tex.sprint(math.pi)}$''
in your document. See how easy it is to mix \tex and Lua?

Actually, there are a few gotchas. Let's look at the Lua to \tex way first,
it's the simplest (since it's more Lua than \tex). If you look at the \luatex
manual, you'll see there is another function with a simpler name, ©tex.print©,
to pass information this way. It works by virtually inserting a full line into
your \tex source, whose contents are its argument. In case you didn't know,
\tex does many nasty\footnote{Okay, these are usually nice and helpful
  actions, but in this case they are most probably unexpected so I call them
  nasty.} things with full lines of the source:
ignoring spaces at the beginning and end of line and appending an end-of-line
character. Most of the time, you don't want this to happen, so I recommend
using ©tex.sprint© which virtually inserts its argument in the current line,
so is much more predictable.

If you're enough of a \tex{}nician to know about catcodes, you'll be happy to
know that ©tex.print© and its variants give you nearly full control over the
catcodes used for tokenizing the argument, since you can specify a catcode
table as the first argument. You'll probably want to learn about catcode
tables (currently~2.7.6 in the \luatex manual) before you're fully happy. If
you don't know about catcodes, just skip this paragraph.\footnote{Erf, too
  late, you already read it.}

\medskip

Let's look at ©\directlua© now. To get an idea about how it works, imagine that
it's a ©\write© command, but it writes only to a virtual file and immediately
arranges for this file to be fed to the Lua interpreter. On the Lua end, the
consequence is that each argument of a ©\directlua© command has its own scope:
local variables from one will not be visible to the other. (Which is rather
sane, but always good to know.)

Now, the major gotcha is that before being fed to the Lua interpreter, the
argument is first read and tokenised by \tex, then fully expanded and turned
back into a plain string. Being read by \tex has several consequences. One of
them is that end of lines are turned into spaces, so the Lua interpreter sees
only one (long) line of input. Since Lua is a free-form language, it doesn't
usually matter, but it does if you use comments:
\begin{Verbatim}
  \directlua{a_function()
    -- a comment
    another_function()}
\end{Verbatim}
won't do what you probably expect: ©another_function()© will be seen as part
of the comment (it's only one line, remember).

Another consequence of being read by \tex is that successive spaces are merged
into one space, and \tex comments are discarded. So, here is a surprisingly
correct version of the previous example.
\begin{Verbatim}
  \directlua{a_function()
    % a comment
    another_function()}
\end{Verbatim}
It is also worth noticing that, since the argument basically is inside a
©\write©, it's in expansion-only context. If you don't know what it means, let
me say that expansion issues are mostly what makes \tex programming so
difficult to avoid expanding further on that matter.

\medskip

I'm sorry if the last three paragraphs were a bit \tex{}nical in nature but
I thought you had to know. To reward you for staying with me, here is a
debugging trick. Put the following code near the beginning of your document:
\begin{Verbatim}
  \newwrite\luadebug
  \immediate\openout\luadebug luadebug.lua
  \AtEndDocument{\immediate\closeout\luadebug}
  \newcommand\directluadebug{\immediate\write\luadebug}
\end{Verbatim}
Then, when you have a hard time understanding why a particular call to
©\directlua© doesn't do what you expect, replace this instance of the command
with ©\directluadebug©, compile as usual and look in the file
\file{luadebug.lua} produced what the Lua interpreter actually read.

The \pk{luacode} package provides commands and environments that help to
varying degrees with some of these problems. Hoewever, for everything but
trivial pieces of Lua code, it is wiser to use an external file containing
only Lua code defining functions, then load it and use its functions. For
example:
\begin{Verbatim}
  \directlua{dofile("my-lua-functions.lua")}
  \newcommand*{\greatmacro}[2]{%
    \directlua{my_great_function("\luatexluaescapestring{#1}", #2)}}
\end{Verbatim}
The example assumes that ©my_great_function© is defined in
©my-lua-functions.lua© and takes a string and a number as arguments. Notice
that we carefully use the ©\luatexluaescapestring© primitives on the string
argument to escape any backslash or double-quote it might contain and which
would confuse the Lua parser.\footnote{If you ever used SQL then the concept
  of escaping strings is hopefully not new to you.}

\medskip

That's all concerning Lua in \tex. If you're wondering why
©\luatexluaescapestring© has such a long and silly name, you might want to
read the next section.

\subsection{Other things you should know}\label{things}

Just in case it isn't obvious, the \luatex manual, \file{luatexref-t.pdf}, is
a great source of information about \luatex and you'll probably want to
consult it at some point (though it is a bit arid and technical).

It is important to know that the name of the new primitives of \luatex as you
read them in the manual are not the actual names you'll be able to use in
\lualatex. To prevent clashes with existing macro names, all new primitives
have been prefixed with ©\luatex© unless they already start with it, so
©\luaescapestring© becomes ©\luatexluaescapetring© while ©\luatexversion©
remains ©\luatexversion©. The rationale is detailed in section~\ref{formats}.

\medskip

Oh, and by the way, did I mention that \luatex is in beta and version 1.0 is
expected in spring 2014? You can learn more on the roadmap page of
\href{http://luatex.org/}{the \luatex site}. Stable betas are released
regularly and are included in \texlive since 2008 and \miktex since 2.9.

Not surprisingly, support for \luatex in \latex is shiny new, which means it
may be full of (shiny) bugs, and things may change at any point. You might
want to keep your \tex distribution very up-to-date\footnote{For \texlive,
  consider using the complementary
  \href{http://tlcontrib.metatex.org/} {tlcontrib} repository.} and also avoid
using \lualatex for critical documents at least for some time.

As a general rule, this guide documents things as they are at the time it is
written or updated, without keeping track of changes. Hopefully, you'll update
your distribution as a whole so that you always get matching versions of this
guide and the packages, formats and engine it describes.

\section{Essential packages and practices}\label{essential}

This section presents the packages you'll probably want to always load as a
user, or that you should absolutely know about as a developer.

\subsection{User-level}

\pkdesc{fontspec}{\WSPR}{\xetex, \luatex}{\latex}{%
  macros/latex/contrib/fontspec/}[https://github.com/wspr/fontspec/]
Nice interface to font management, well-integrated in to the \latex font
selection scheme. Already presented in the previous section.

\pkdesc{polyglossia}{\FC \& \AR}{\xetex, \luatex}{\latex}{%
  macros/latex/contrib/polyglossia/}[https://github.com/reutenauer/polyglossia/]
A simple and modern replacement for Babel, working in synergy with \pk{fontspec}.

\subsection{Developer-level}

\subsubsection{Naming conventions}

On the \tex end, control sequences starting with ©\luatex© are reserved for
primitives. It is strongly recommended that you do \emph{not} define any such
control sequence, to prevent name clashes with future versions of \luatex. If
you want to a emphasize that a macro is specific to \luatex, we recommend that
you use the ©\lua© prefix (without a following ©tex©) instead. It is okay to
use the ©\luatex@© prefix for internal macros, since primitive names never
contain ©@©, but it might be confusing. Moreover, you're already using a
unique prefix for internal macros in all of your packages, aren't you?

On the Lua end, please keep the global namespace as clean as possible. That
is, use a table ©mypackage© and put all your public functions and objects in
this table. You might want to avoid using Lua's deprecated \code{module()}.
Other strategies for Lua module management are discussed in
\href{http://www.lua.org/pil/15.html} {chapter~15 of \emph{Programming in
    Lua}}, and examples are given in \file{luatexbase-modutils.pdf}.  Also, it
is a good and necessary practice to use ©local© for your internal variables and
functions. Finally, to avoid clashes with future versions of \luatex, it is
necessary to avoid modifying the namespaces of \luatex's default libraries.

\subsubsection{Engine and mode detection}\label{detect}

Various packages allow to detect the engine currently processing the document.

\pkdesc{ifluatex}{\HO}{all}{\latex, Plain}{%
  macros/latex/contrib/oberdiek/}
Provides ©\ifluatex© and makes sure ©\luatexversion© is available.

\pkdesc{iftex}{\VK}{all}{\latex, Plain}{%
  macros/latex/contrib/iftex/}[http://bitbucket.org/vafa/iftex]
Provides ©\ifPDFTeX©, ©\ifXeTeX©, ©\ifLuaTeX© and corresponding ©\Require©
commands.

\pkdesc{expl3}{The \LaTeX3 Project}{all}{\latex}{%
  macros/latex/contrib/expl3/}[http://www.latex-project.org/code.html]
Amongst \emph{many} other things, provides ©\luatex_if_engine:TF©,
©\xetex_if_engine:TF© and their variants.

\pkdesc{ifpdf}{\HO}{all}{\latex, Plain}{%
  macros/latex/contrib/oberdiek/}
Provides ©\ifpdf© switch. \luatex, like \pdftex, can produce either PDF or DVI
output; the later is not very useful with \luatex as it doesn't support any
advanced feature such as Unicode and modern font formats. The ©\ifpdf© switch
is true if and only if you are running \pdftex-or-\luatex in PDF mode (note
that this doesn't include \xetex, whose support for PDF is different).

\subsubsection{Basic resources}

\pkdesc{luatexbase}{\ER, \MPG \& \PHG}{\luatex}{\latex, Plain}{%
  macros/luatex/generic/luatexbase/}[https://github.com/lualatex/luatexbase]
The Plain and \latex formats provide macros to manage \tex basic resources,
such as count or box registers. \luatex introduces new resources that need to
be shared gracefully by packages. This package provides the basic tools to
manage: the extended conventional \tex resources, catcode tables, attributes,
callbacks, Lua modules loading and identification. It also provides basic
tools to handle a few compatibility issues with older version of \luatex.

\note{Warning} This package is currently in conflict with the \pk{luatex}
package, since they both do almost the same thing. The respective package
authors are well aware of this situation and plan to somehow merge the two
packages in the near future, though the timeline is not clear.

\pkdesc{luatex}{\HO}{\luatex}{\latex, Plain}{%
  macros/latex/contrib/oberdiek/}
See the description of \pk{luatexbase} above. This package provides the same
core features except for callback management and Lua module identification.

\pkdesc{lualibs}{\ER \& \PHG}{\luatex}{Lua}{%
  macros/luatex/generic/lualibs/}[https://github.com/lualatex/lualibs]
Collection of Lua libraries and additions to the standard libraries; mostly
derived from the \context libraries. If you need a basic function that Lua
doesn't provide, check this package before rolling your own implementation.

\subsubsection{Font internals}\label{fontint}

Those packages are loaded by \pk{fontspec} to handle some low-level font and
encoding issues. A normal user should only use \pk{fontspec}, but a developer
may need to know about them.

\pkdesc{luaotfload}{\ER, \KH \& \PHG}{\luatex}{\latex, Plain}{%
  macros/luatex/generic/luaotfload/}[https://github.com/lualatex/luaotfload]
Low-level OpenType font loading, adapted from the generic subset of \context.
Basically, it uses the ©fontloader© Lua library and the appropriate callbacks
to implement a syntax for the ©\font© primitive very similar to that of \xetex
and implement the corresponding font features. It also handles a font database
for transparent access to fonts from the system and the \tex distribution
either by family name or by file name, as well as a font cache for faster
loading.

\pkdesc{euenc}{\WSPR, \ER \& \KH}{\xetex, \luatex}{\latex}{%
  macros/latex/contrib/euenc/}[https://github.com/wspr/euenc]
Implements the EUx Unicode font encodings for \latex's \pf{fontenc} system.
Currently, \xelatex is using ©EU1© and \lualatex is using ©EU2©. Includes
definitions (\file{fd} files) for Latin Modern, the default font loaded by
\pk{fontspec}.

To be precise, \pf{euenc} merely declares the encoding, but
doesn't provide definitions for LICR macros; this is done by loading
\pk{xunicode} with ©\UTFencname© defined to ©EU1© or ©EU2©, which
\pk{fontspec} does. The actual encodings are the same, but it is useful to
have distinct names so that different \file{fd} files can be used according to
the engine (which is actually the case with Latin Modern).

\section{Other packages}\label{other}

Note that the packages are not listed in any particular order.

\subsection{User-level}

\pkdesc{luatextra}{\ER \& \MPG}{\luatex}{\latex}{%
  macros/luatex/latex/luatextra/}[https://github.com/lualatex/luatextra]
Loads usual packages, currently \pk{fontspec}, \pk{luacode}, \pf{metalogo}
(commands for logos, including ©\LuaTeX© and ©\LuaLaTeX©), \pk{luatexbase},
\pk{lualibs}, \pf{fixltx2e} (fixes and enhancements for the \latex core).

\pkdesc{luacode}{\MPG}{\luatex}{\latex}{%
  macros/luatex/latex/luacode/}[https://github.com/lualatex/luacode]
Provides commands and macros that help including Lua code in a \tex source,
especially concerning special characters.

\pkdesc{luainputenc}{\ER \& \MPG}{\luatex, \xetex, \pdftex}{\latex}{%
  macros/luatex/latex/luainputenc/}[https://github.com/lualatex/luainputenc]
Helps compiling documents relying on legacy encodings (either in the source or
with the fonts). Already presented in the introduction. When running \xetex,
just loads \pf{xetex-inputenc}; under \pdftex, loads the standard
\pf{inputenc}.

\pkdesc{luamplib}{\HH, \Taco \& \PHG}{\luatex}{\latex, Plain}{%
  macros/luatex/generic/luamplib/}[https://github.com/lualatex/luamplib]
Provides a nice interface for the ©mplib© Lua library that embeds metapost in
\luatex.

\pkdesc{luacolor}{\HO}{\luatex}{\latex}{%
  macros/latex/contrib/oberdiek/}
Changes low-level color implementation to use \luatex attributes in place of
whatsits. This makes the implementation more robust and fixes strange bugs
such as wrong alignment when ©\color© happens at the beginning of a ©\vbox©.

\subsection{Developer-level}

\pkdesc{pdftexcmds}{\HO}{\luatex, \pdftex, \xetex}{\latex, Plain}{%
  macros/latex/contrib/oberdiek/}
Though \luatex is mostly a superset of \pdftex, a few utility primitives were
removed (those that are sort of superseded by Lua) or renamed. This package
provides them with consistent names across engines, including \xetex which
recently implemented some of these primitives, such as ©\strcmp©.

\pkdesc{magicnum}{\HO}{\luatex, \pdftex, \xetex}{\latex, Plain}{%
  macros/latex/contrib/oberdiek/}
Provides hierarchical access to ``magic numbers'' such as catcodes, group
types, etc. used internally by \tex and its successors. Under \luatex, a more
efficient implementation is used and a Lua interface is provided.

\pkdesc{lua-alt-getopt}{Aleksey Cheusov}{\cmd{texlua}}{command-line}{%
  support/lua/lua-alt-getopt}[https://github.com/LuaDist/alt-getopt]
Command-line option parser, mostly compatible with POSIX and GNU getopt, to be
used in command-line Lua scripts such as \cmd{mkluatexfontdb} from
\pk{luaotfload}.

\section{The \cmd{luatex} and \cmd{lualatex} formats}\label{formats}

This section is for developers and curious users only; normal users can safely
skip it. The following information apply to \texlive 2010, and most likely to
\miktex 2.9 too, though I didn't actually check. Earlier versions of \texlive
had slightly different and less complete arrangements.

\para{Primitive names}
As mentioned in section~\ref{things}, the names of the \luatex-specific
primitives are not the same in the \cmd{lualatex} format as in the \luatex
manual. In the \cmd{luatex} format (that is, \luatex with the Plain format),
primitives are available with their natural name, but also with the prefixed
name, in order to ease development of generic packages.

The rationale, copy-pasted from the file \file{lualatexiniconfig.tex} that
implements this for the \cmd{lualatex} format, is:

\begin{myquote}
  \begin{enumerate}
    \item All current macro packages run smoothly on top of pdf(e)TeX, so
      those primitives are left untouched.
    \item Other non-TeX82 primitives in \luatex may cause name clashes with
      existing macros in macro packages, especially when they use very
      ``natural'' names such as ©\outputbox©, ©\mathstyle© etc. Such a
      probability for name clashes is undesirable, since the most existing
      LaTeX documents that run without change under \luatex, the better.
    \item The \luatex team doesn't want to apply a systematic prefixing policy,
      but kindly provided a tool allowing prefixes to be applied. So we chose
      to use it.  Previously, we even disabled the extra primitives, but now
      we feel it's better to enable them with systematic prefixing, in order
      to avoid that each and every macro package (or user) enables them with
      various and inconsistent prefixes (including the empty prefix).
    \item The ©luatex© prefix was chosen since it is already used as a prefix
      for some primitives, such as ©\luatexversion©: this way, those
      primitives don't end up with a double prefix (for details, see
      ©tex.enableprimitives© in the \luatex manual).
    \item The ©\directlua© primitive is provided both with its natural name
      (allowing easy detection of \luatex) and a prefixed version
      ©\luatexdirectlua© (for consistency with ©\luatexlatelua©).
    \item Various remarks:
      \begin{itemize}
        \item The obvious drawback of such a prefixing policy is that the
          names used by \latex or generic macro writer won't match the names
          used in the manual.  We hope this is compensated by the gain in
          backwards compatibility.
        \item All primitives dealing with Unicode math already begin with ©\U©,
          and maybe will match the names of \xetex primitives some day, so
          maybe prefixing was not necessary/desirable for them. However, we
          tried to make the prefixing rule as simple as possible, so that
          the previous point doesn't get even worse.
        \item Maybe some day we'll feel it's better to provide all primitives
          without prefixing. If this happens, it will be easy to add the
          unprefixed primitives in the format while keeping the prefixed names
          for compatibility. It wouldn't work the other way round; i.e.,
          belatedly realizing that we should not provide the unprefixed
          primitives would then break any \luatex-specific macro packages
          that had been written.
      \end{itemize}
  \end{enumerate}
\end{myquote}

\para{\cs{jobname}}[jobname]
The \latex kernel (and a lot of packages) use constructs like
©\input\jobname.aux© for various purposes. When ©\jobname© contains spaces,
this doesn't do the right thing, since the argument of ©\input© ends at the
first space. To work around this, \pdftex automagically quotes ©\jobname© when
needed, but \luatex doesn't for some reason. A nearly complete workaround is
included in \latex-based (as opposed to Plain-based) \luatex formats.

It doesn't work, however, if \luatex is invoked as ©lualatex '\input name'©,
as opposed to the more usual ©lualatex name©. To work around this
limitation of the workaround included in the format, specifying a jobname
explicitly, as in ©lualatex jobname=name '\input name'©. Or even better, just
don't use spaces in the names of your \tex files.

For more details, see
\href{http://www.ntg.nl/pipermail/dev-luatex/2009-April/002549.html}{this old
  thread} and
\href{http://tug.org/pipermail/luatex/2010-August/001986.html}{this newer one}
on the \luatex mailing lists, and \file{lualatexquotejobname.tex} for the
implementation of the workaround.

\para{hyphenation}
\luatex offers dynamic loading for hyphenation patterns. The support for this in
\pf{babel} and \pf{polyglossia} appeared only on \texlive 2013, but should
work well since then.

Documentation and implementation details are included in
\file{luatex-hyphen.pdf}. The sources are part of the
\href{http://tug.org/tex-hyphen/}{texhyphen project}.

\para{codes}
The engine itself does not set ©\catcode©s, ©\lccode©s, etc. for non-ASCII
characters. Correct ©\lccode©s in particular are essential for hyphenation to
work. Formats for \luatex now include \file{luatex-unicode-letters.tex}, a
modified version of \file{unicode-letters.tex} from the \xetex distribution,
that takes care of settings these values in accordance with the Unicode
standard.

This was added after \texlive 2010 went out, so you are strongly advised to
update your installation if you want to enjoy proper hyphenation for non-ASCII
text.


\section{Things that just work, partially work, or don't work (yet)}
\label{workornot}

\subsection{Just working}\label{working}

\para{Unicode}
Conventional \latex offers some level of support for UTF-8 in input files.
However, at a low level, non-ASCII characters are not atomic in this case:
they consist of several elementary pieces (known as \emph{tokens} to
\tex{}nicians). Hence, some packages that scan text character by character or
do other atomic operations on characters (such as changing their catcodes)
often have problems with UTF-8 in conventional \latex. For example, you can't
use any non-ASCII character for short verbatim with \pf{fancyvrb}, etc.

The good news is, with \lualatex, some of these package's features start
working on arbitrary Unicode characters without needing to modify the package.
The bad news is, this isn't always true. See the next section for details.

\subsection{Partially working}\label{partial}

\para{microtype}
Package \pf{microtype} has limited support for \luatex: more precisely, as of
version 2.5 2013/03/13, protrusion and expansion are available and activated
by default in PDF mode, but kerning, spacing and tracking are not supported
(see table~1 in section~3.1 of \file{microtype.pdf}).

On the other hand, \pk{luaotfload}, loaded by \pk{fontspec}, supports a lot of
microtypographic features. So the only problem is the lack of a unified
interface.

\para{xunicode}
Package \pf{xunicode}'s main feature is to ensure that the usual control
sequences for non-ASCII characters (such as ©\'e©) do the right thing in a
Unicode context. It could \emph{probably} work with \luatex, but explicitly
checks for \xetex only. However, \pk{fontspec} uses a trick to load it anyway.
So, you can't load it explicitly, but you don't need to, since \pk{fontspec}
already took care of it.

\para{encodings}
As mentioned in the above section, a few things that were problematic with
UTF-8 on conventional \latex spontaneously works, but not always. For example,
with the \pf{listings} package on \lualatex, you may use only characters below
256 (that is, characters from the Latin-1 set), inside your listings (but of
course the full Unicode range is still available in the rest of your
document).

\para{metrics}
This item isn't exactly about working or not working, but rather about not
working in exactly the same way as \pdftex or \xetex: you may observe minor
differences in the layout and hyphenation of your text.

They may be due variations between two versions of the same font used by the
various engines, adjustments made to the hyphenation, ligaturing or kerning
algorithms (or example, the first word of a paragraph, as well as words
containing different fonts, can now be hyphenated), or differences in the
hyphenation patterns used (patterns used by \pdftex are basically frozen, but
\luatex and \xetex use newer version for some languages) for this language.

If you ever observe a major difference between pdf\latex and \lualatex with
the same fonts, it is not at all unlikely that a bug in \luatex\footnote{For
  example, \luatex 0.60 had a bug that prevented any hyphenation after a
  \code{-{}-{}-} ligature until the end of the paragraph.} or in the font is
involved. As usual, make sure your distribution is up-to-date before reporting
such a problem.

\para{babel}
Mostly working mostly without problems for Latin languages. For other
languages, your mileage may vary. Even for Latin languages, encoding-related
problems my happen.

\para{polyglossia}
A more modern, but less complete, package for multilingual support,
\pf{polyglossia}, is also available and should be preffered, though it does
not support all \pf{babel} features yet.

\subsection{Not working (yet)}\label{notworking}

\para{old encodings}[oldenco] Packages playing with input (source files) or
output (fonts) encodings are very likely to break with \luatex. This includes
\pf{inputenc}, \pf{fontenc}, \pf{textcomp}, and probably most classical font
packages such as \pf{mathptmx} or \pf{fourier}. The good news
is, Unicode is a more powerful way to handle encoding problems that old
packages were trying to solve, so you most likely don't need these packages
anyway. However, not everything is already ported to the shiny new world of
Unicode, so you may have a more limited (or just different) set of choices
available for some time (this is especially true for fonts).

\para{spaces} Spaces in file names are not really well supported in the \tex
world in general. This doesn't really get better with \luatex. Also, due to
tricky reasons, things may be worse if you have spaces in the name of your main
\tex file \emph{and} don't invoke \luatex in the usual way. If you do
invoke it in the usual way, everything should work, and I won't tell you what
the unusual invocation looks like. Otherwise, read the point about
\pararef{jobname} in section~\ref{formats} for a workaround and technical
details. Or even better, don't use spaces in the names of your \tex files.

\end{document}

% vim: spell spelllang=en
