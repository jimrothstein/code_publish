% example of latex | pandoc | sections
% USAGE: :!pandoc -f latex -t pdf -o - | zathura -
% USAGE:  (fails):!lualatex %; zathura
\documentclass{article}
\usepackage{amsmath}
\usepackage{lipsum}

\title{Sections and Chapters}
\author{Gubert Farnsworth}
\date{\today}

\begin{document}
\url{https://milde.users.sourceforge.net/LUCR/Math/mathpackages/amssymb-symbols.pdf}


\section {Dedekind Cut}

$$
\href{https://en.wikipedia.org/wiki/Dedekind_cut}{https://en.wikipedia.org/wiki/Dedekind_cut}
$$


His definition of infinite set:   A set with proper subset of same cardinality
is infiinte.

$$
Z \subset 2Z
Z \subset Q \subset algebraic reals
$$


\section { Normal }
A normal subgroup of B

$$
A	\vartriangleleft B
$$

\begin{align*}
%   Note:  use of double backslash inside environment
G/N \\  
a \sim b \\
\mathbb{N} / n\mathbb{N}
\end{align*}

\begin{align}
%   Note:  use of double backslash inside environment
G/N \\  
a \sim b \\
\mathbb{N} / n\mathbb{N}
\end{align}

awful
\begin{equation}
%   Note:  use of double backslash inside environment
G/N \\  
a \sim b \\
\mathbb{N} / n\mathbb{N}
\end{equation}


\section {Computable Number (Turing)}
Construct digit-by-digit ie  ~ $|\mathbb{Z}|$ \
Each digit takes a finite number of steps to compute.


All possible computer programs ~ $|\mathbb{Z}|$

Consider every T/F question, 1 if T, 0 if F (Borel, NOT a computable number?)
=0.1101$\dots$




\end{document}
