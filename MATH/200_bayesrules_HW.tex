\documentclass{article}
\usepackage{ulem}
\usepackage{xcolor}
\usepackage{lipsum}
\usepackage{amssymb}
\newcommand{\dg}{^\circ}
% \newcommand{\prob}{P(A $\mid$ B)}

\usepackage{amsthm}

\theoremstyle{definition}
\newtheorem{definition}{Definition}[section]

\theoremstyle{remark}
\newtheorem*{remark}{Remark}

\begin{document}

% cmd name, number args, what it does
\newcommand{\bb}[1]{\mathbb{#1}}
\newcommand{\prob}[2]{P({#1} $\mid$ {#2})}
% this is to highlight words that are being defined and enter.
\newcommand{\define}[1]{\textbf{#1}}

 
As of \today

new dg command:  $\dg$



prob \prob{A}{Z}

Complex command:
$$
Complex: \bb{C}
$$

%
%   pdflatex %  works !

% :!lualatex % 
%   !pandoc % -t latex -o out.pdf 

%   pandoc does it, but complains
%   !pandoc % -f latex -t pdf -o - | zathura -
%>% comment


% \tableofcontents


\begin{verbatim}
file <- "/home/jim/code/publish_project/MATH/100_math_examles.md"

\end{verbatim}
PURPOSE:	Collect examples of math/latex here:  vectors, equations, align,
symbols etc.

<!--	This is comment to pandoc

%  comment in .tex, but not comment to pandoc

-->


```
Exercise 2.1 (Comparing the prior and posterior) For each scenario below, you’re given a pair of events, A
and B. Explain what you believe to be the relationship between the posterior and prior probabilities of B: P(B|A)>P(B) or P(B|A)<P(B)
.

    A
(a)
A = you just finished reading Lambda Literary Award-winning author Nicole Dennis-Benn’s first novel, and you enjoyed it! 

B = you will also enjoy Benn’s newest novel.

A = it’s 0 degrees Fahrenheit in Minnesota on a January day. 
B = it will be 60 degrees tomorrow.

A
= the authors only got 3 hours of sleep last night. B
= the authors make several typos in their writing today.
A
= your friend includes three hashtags in their tweet. B = the tweet gets retweeted. 
```


We are told A, and asked to consider B:  P(B | A)
Prior is A.
P(B | A) =  L(A | B)


(b) \\
Prior is P(A)  Posterior is P(B | A )\\
Of all possible outcomes, B, seems that P(B | A) is quite low.\\

\section{Terms $\&$ Definitions}
Y outcomes random variable, based upon X inputs, model parameters \\
Because of error,  we refer to P(Y $\mid$ X,  $\pi$ ), where $\pi$ is model parameter\\

Joint
Conditional
Event
Sample Space
Partition
Marginal

Categorical vs Binary vs Discrete variable

\begin{definition}[test]  
$$
(\Omega, \mathcal{F}, P ) \\
$$
Consider all the subsets of $\Omega$\\

Then $\mathcal{F}$ contains family of such such subsets, containng  $\emptyset, \Omega,$ 

and if $x \in \mathcal{F}$ then $x^\complement \in \mathcal{F}$

For example, $\Omega$ is  unit square and consider simple curve st $\in \mathcal{F}$ .  
though  could be $\mathcal{P}(\Omega)$ \\

Elements of $\mathcal{F} \sigma$ algebra (if countable) or algebra if finite and
maps between subsets of $\Omega$ \\
Rules:  
\end{definition}

\begin{definition}[Probability Triple]  
$$
$$
$$
$$ 
\end{definition}




\section{linear regression}


\end{document}

vim:linebreak:nospell:nowrap:cul tw=78 fo=ntl foldcolumn=3 cc=+1 ft=tex
